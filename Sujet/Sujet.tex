\documentclass{article}
\usepackage[utf8]{inputenc}

\title{Proposition de sujet de mémoire}
\author{Aboubacar Sidiki CONDE}
\date{\today}

\begin{document}

\maketitle


\section{Titre}
Modernisation des processus de suivi de production et de maintenance par le développement d'applications web sur mesure : étude de cas chez Jeumont Electric
\section{Problèmatique}
Dans un contexte industriel où le suivi de production et la gestion de la maintenance des équipements et machines électriques sont des enjeux critiques, les outils traditionnels comme les fichiers Excel révèlent leurs limites en termes de flexibilité, de collaboration et de scalabilité. Comment la transformation numérique, à travers le développement d’applications web sur mesure, peut-elle moderniser et optimiser la gestion des processus ainsi que la maintenance des systèmes chez Jeumont Electric ? Quels sont les défis techniques et organisationnels liés à la migration d’anciens systèmes de suivi de production vers des solutions web modernes, et comment ces nouvelles technologies améliorent-elles l’efficacité opérationnelle, la réactivité et la productivité globale de l’entreprise?
dddddddd
\end{document}